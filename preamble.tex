%%%%%%%%%%%%%%%%%%%%%%%%%%%%
%%   Zusaetzliche Pakete  %%
%%%%%%%%%%%%%%%%%%%%%%%%%%%%
%Plotbar
\usepackage[english]{babel}
\usepackage{pgfplots}
\pgfplotsset{compat=1.17}
\usepackage{pgfplotstable}
\usepackage{multirow}
\usepackage{colorprofiles}

\usepackage[hidelinks]{hyperref}

\usepackage{xspace}

\usepackage{pdfpages}

\usepackage{acro}
\usepackage{pdfcomment}
\usepackage{longtable}
\acsetup{
	make-links 		= 	true,
	pdfcomments/use	=	true,
	%remove the section header
	list/heading	=	none,
	list/template	=	longtable % Use the longtable environment for better control
}

% Custom colors
\usepackage{color}
\definecolor{deepblue}{rgb}{0,0,0.8}
\definecolor{deepred}{rgb}{0.8,0,0}
\definecolor{deepgreen}{rgb}{0,0.8,0}
\definecolor{blackbackground}{rgb}{0.1,0.1,0.2}

\usepackage{listings}

% Python style for highlighting
\newcommand\pythonstyle{\lstset{
	language=Python,
	basicstyle=\ttfamily \color{white},
	morekeywords={self},              % Add keywords here
	keywordstyle=\ttb\color{deepblue},
	emph={MyClass,__init__},          % Custom highlighting
	emphstyle=\ttb\color{deepred},
	stringstyle=\color{deepgreen},
	showstringspaces=false,
	breaklines=true,
	backgroundcolor=\color{blackbackground}
}}

% Python environment
\lstnewenvironment{python}[1][]
{
\pythonstyle
\lstset{#1}
}
{}

% Python for external files
\newcommand\pythonexternal[2][]{{
\pythonstyle
\lstinputlisting[#1]{#2}}}

% Python for inline
\newcommand\pythoninline[1]{\colorbox{blackbackground}{\texttt{\color{white}\lstinline!#1!}}}


\renewcommand{\lstlistingname}{Code}

\usepackage{enumerate}
\usepackage{fancyhdr}
\usepackage{a4wide}
\usepackage{graphicx}
\usepackage{titlesec}
\usepackage{enumitem}
\usepackage[table]{xcolor}
\usepackage{colortbl}
\usepackage{caption}
\usepackage{nameref}
\usepackage{tikz}
\usepackage{verbatim}
\usepackage{csquotes}
\usepackage{lmodern} 
\usepackage{booktabs} 
% \hypersetup{%Change later
% 	colorlinks=true,
% 	linkcolor=red,
% 	urlcolor=red,
% 	citecolor=red
% }
\usepackage{array}
\usepackage{flowchart}
\usepackage{makecell,tabularx} %Für Tabellen

\usepackage{siunitx}
\usepackage{amsmath}
\usepackage{setspace} %Zeilenabstand
\renewcommand{\baselinestretch}{1.6}
\usepackage[a4paper, left=2.5cm, right=2.5cm, top=2.5cm, bottom=2.5cm]{geometry} % Randabstand 2,5 cm
\usepackage{tcolorbox}
% Definition der Box mit schwarzem Rahmen
\newtcolorbox{blackframebox}{
	colframe=black, % Rahmenfarbe der Box
	boxrule=1pt, % Rahmendicke
	arc=0pt, % Abrundung der Ecken
	boxsep=5pt, % Abstand zwischen Text und Rahmen
	left=5pt, % Linker Randabstand
	right=5pt, % Rechter Randabstand
	top=5pt, % Oberer Randabstand
	bottom=0pt, % Unterer Randabstand
	%colback=lightgray % Hintergrundfarbe der Box
}
\usepackage[sorting=none, backend=biber, style=numeric]{biblatex}
\usepackage[ddmmyyyy]{datetime}
\usepackage{placeins} %Für FloatBarrier
\usepackage{algorithm}
\usepackage[export]{adjustbox}

\addbibresource{Literature.bib}

\setlength{\parindent}{0em}%Kein Einrücken bei Paragraphen 

%%%%%%%%%%%%%%%%%%%%%%%%%%%%%%
%% Definition der Kopfzeile %%
%%%%%%%%%%%%%%%%%%%%%%%%%%%%%%

\pagestyle{fancy}
\fancyhf{}
\fancyhead[L]{\nouppercase{\leftmark}}
\fancyhead[R]{\kindofthesis~\thesisauthor}
\setlength{\headheight}{15pt}

%%%%%%%%%%%%%%%%%%%%%%%%%%%%%%
%% Definition der Fußzeile %%
%%%%%%%%%%%%%%%%%%%%%%%%%%%%%%

\fancyfoot[C]{Page \thepage}

%%%%%%%%%%%%%%%%%%%%%%%%%%%%%%%%%%%%%%%%%%%%%%%%%%%%%
%%  Definition des Deckblattes und der Titelseite  %%
%%%%%%%%%%%%%%%%%%%%%%%%%%%%%%%%%%%%%%%%%%%%%%%%%%%%%
\newcommand{\CustomTitle}[0]{
	\thispagestyle{empty}
	\begin{figure}
		\begin{minipage}{0.4\linewidth}
			\includegraphics[height=2.0cm,left]{Bilder/LogoLeft.png} 
		\end{minipage} 
		\hfill
		\begin{minipage}{0.4\linewidth}
			\includegraphics[height=2.0cm,right]{Bilder/LogoRight.png}
		\end{minipage}
	\end{figure}
  
  \vspace*{\stretch{1}}
  \begin{center}
    \vspace*{\stretch{0.5}}
    \bfseries\Huge \thesistitle\\

	\vspace*{\stretch{0.5}}
    \bfseries\Huge \kindofthesis\\	 

    \vspace*{\stretch{0.5}}
    \bfseries\large
	von
    \\
    \thesisauthor\\
    \vspace*{\stretch{0.5}}
    \small
  \end{center}

  \vspace*{\stretch{1}}

  \begin{tabular}{l l}
    Working Period: & \thesisstartdate ~- \thesisenddate\\
    Matriculation Number: & \immatrikulationnumber\\
    Course: & \universitycourse\\
    Program: & \studycourse\\
    University: & \university\\ 
    Submission Date: & \thesisenddate\\
    Academic Examiner: & \unicorrector\\
    Company Examiner: & \companycorrector\\
\end{tabular}

 \newpage
}


\titlespacing*{\section}
{0pt}{3.5ex plus 1ex minus .2ex}{.2ex plus .2ex}
\titlespacing*{\subsection}
{0pt}{1.5ex plus 1ex minus .2ex}{.2ex plus .2ex}
\titlespacing*{\subsubsection}
{0pt}{1.5ex plus 1ex minus .2ex}{.2ex plus .2ex}

\setcounter{tocdepth}{2}%Ebenen im Inhaltsverzeichnis
\setcounter{secnumdepth}{3}%Ebenen im Inhaltsverzeichnis

%%%%%%%%%%%%%%%%%%%%%%%%%%%%
%%  Beginn des Dokuments  %%
%%%%%%%%%%%%%%%%%%%%%%%%%%%%